\documentclass[xcolor]{beamer}
\usepackage[utf8]{inputenc}
\usepackage[T1]{fontenc}
\usepackage{babel}
\usepackage{csquotes}
\usepackage[sorting=none,giveninits=true]{biblatex}
\addbibresource{Bibliography.bib}
\usepackage{float}
\usepackage{xcolor}  % kolory motywu
\usepackage{tikz}
\usetikzlibrary{angles}
\usetikzlibrary{quotes}
\usetikzlibrary{decorations.pathreplacing}
\usetikzlibrary{calligraphy}
\usetikzlibrary{arrows.meta}
\usetikzlibrary{calc}
\usepackage{pgfplots}
\usepackage{pgfplotstable}
\pgfplotsset{compat=1.9}
\usepackage{amsmath}  % równania
\usepackage{amssymb}
\usepackage{bbold}
\usepackage{physics2}  % pochodne, macierze itp
\usephysicsmodule{ab}
\usephysicsmodule{diagmat}
\usephysicsmodule{xmat}
\usephysicsmodule{nabla.legacy}
\usephysicsmodule{op.legacy}
\usefonttheme[onlymath]{serif}
%\usepackage{package}
\makeatletter
%\newcommand\vb[1]{\@ifstar\boldsymbol\mathbf{#1}}
\newcommand\vb[1]{\boldsymbol{#1}}
%\newcommand\vb[1]{\@ifstar\boldsymbol\mathbf{#1}}
\newcommand\va[1]{\@ifstar{\vec{#1}}{\vec{\mathrm{#1}}}}
\newcommand\vu[1]{%
\@ifstar{\hat{\boldsymbol{#1}}}{\hat{\boldsymbol{#1}}}}
\makeatother
\usepackage{fixdif, derivative}  % pochodne
\usepackage[version=4]{mhchem}
\usepackage{siunitx}
\usepackage{booktabs}
\usepackage{makecell}
\title{Electron Paramagnetic Resonance \\ and \\ Nuclear Magnetic Resonance}
\author{Rafał Staroszczyk}
\date{}
\usetheme{Hannover}
\usecolortheme{spruce}

\DeclareMathOperator{\Col}{Col}
\DeclareMathOperator{\Nul}{Nul}
\DeclareMathOperator{\arctg}{arctg}
\DeclareMathOperator{\tgh}{tgh}


\setlength{\abovedisplayskip}{0pt}
\setlength{\belowdisplayskip}{0pt}
\setlength{\abovedisplayshortskip}{0pt}
\setlength{\belowdisplayshortskip}{0pt}

\newcommand{\inv}[1]{\frac{1}{#1}}

\begin{document}
\maketitle
\begin{frame}{Table of Contents}
\tableofcontents
\end{frame}

\section{Magnetic resonance spectroscopy}
\begin{frame}{Magnetic resonance spectroscopy}
\begin{block}{Common characteristics}
	\begin{enumerate}
		\item both methods utilize magnetic dipole moment related to spin
		\item external constant magnetic field splits energy levels
		\item oscillating magnetic field causes transitions
	\end{enumerate}
\end{block}

\begin{block}{Differences}
	\begin{enumerate}
		\item EPR uses electron spin, NMR uses nucleus spin
		\item electron's magnetic dipole moment is $\approx 2000$ times larger than proton's
		\item EPR requires $\qty{1}{\tesla}$, NMR requires $\qty{20}{\tesla}$
		\item EPR operates on microwave frequencies, NMR on radio frequencies
	\end{enumerate}
\end{block}
\end{frame}

\begin{frame}
	Particles with angular momentum are magnetic dipoles with moment:
	\begin{align*}
		\vb{\mu}_{J} &= \gamma_{J} \vb{J} & \vab{\vb{J}} &= \hbar\sqrt{J\pab{J+1}} & J_{z} &= \hbar m_{J}.
	\end{align*}
	For electrons we can separate angular and spin momenta:
	\begin{align*}
		\vb{\mu}_{l} &= \frac{\mu_B}{\hbar} \vb{L} \\
		\vb{\mu}_{s} &= g_e\frac{\mu_B}{\hbar} \vb{S} \\
		\mu_B &= \frac{e\hbar}{2 m_e} \approx \qty{5.788e-5}{\eV.\tesla^{-1}}.
	\end{align*}
	\begin{align*}
		U_{I} &= -g_{I}\mu_{N}m_{I}B_{0} & U_{S} &= g_{e}\mu_{B}m_{S}B_{0} \\
		U_{I}^{\pab{2}} - U_{I}^{\pab{1}} &= g_{I}\mu_{N}B_{0} = h\nu_{p} & U_{S}^{\pab{2}} - U_{S}^{\pab{1}} &= g_{S}\mu_{B}B_{0} = h\nu_{e}
	\end{align*}
\end{frame}

\begin{frame}
Inserting sample in a strong magnetic field splits the energy levels due to Zeeman effect.
	\begin{figure}[H]
		\centering
		\includegraphics[width=\textwidth]{"pigon_1158_1159_p".png}
		\caption{Energy splitting in magnetic field \cite{pigon}}
	\end{figure}
\end{frame}

\section{Nuclear Magnetic Resonance}
\begin{frame}{Continuous Wave Spectroscopy}
	\begin{figure}[H]
		\centering
		\includegraphics[width=0.5\textwidth]{"pigon_1160_p".png}
		\caption{Bloch NMR spectrometer \cite{pigon}}
	\end{figure}
	\begin{align*}
		\nu &= \qtyrange[range-phrase=-]{100}{800}{\MHz} \\
		B &= \qtyrange[range-phrase=-]{10}{20}{\tesla}
	\end{align*}
\end{frame}

\section{Electron Paramagnetic Resonance}
\begin{frame}{Wavebands}
	\begin{table}[H]
		\centering
		\begin{tabular}{
		>{\centering}p{2cm}
		S[table-column-width=2cm]
		S[table-column-width=2cm]
		S[table-column-width=2cm]}
		\toprule
		{Band} & {\makecell{Frequency \\ $\bab{\unit{\GHz}}$}} & {\makecell{Wavelength \\ $\bab{\unit{\cm}}$}} & {\makecell{Magnetic \\ induction $\bab{\unit{\tesla}}$}} \\
		\midrule
		S & 3.0 & 10    & 0.107 \\
		X & 9.5 & 3.15  & 0.339 \\
		K & 23  & 1.30  & 0.82  \\
		Q & 25  & 0.86  & 1.25  \\
		W & 95  & 0.315 & 3.3   \\
		\bottomrule
		\end{tabular}
		\caption{EPR wavebands \cite{pigon}}
	\end{table}
\end{frame}

\begin{frame}{EPR Spectrometer}
	\begin{figure}[H]
		\centering
		\includegraphics[width=\textwidth]{"pigon_1161_p".png}
		\caption{EPR Spectrometer \cite{pigon}}
	\end{figure}
\end{frame}

\begin{frame}{Bibliography}
	\printbibliography
\end{frame}
\end{document}